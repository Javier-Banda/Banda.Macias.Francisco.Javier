\documentclass[11pt,a4paper]{article}
\usepackage{graphicx}
\usepackage{amsmath}
\usepackage{amssymb}
\usepackage{mathrsfs}
\usepackage{cancel}


\begin{document}
\begin{center}
\textbf{Tarea 5}\\
EV 2-4 Giro de un motor en corriente directa
\end{center}

\begin{center}
Banda Macias Francisco Javier\\
15-Octubre-2019\\
Universidad Politecnica de La Zona Metropolitana de Guadalajara
\end{center}
\section{Giro de un motor en corriente directa}
“Giro de motor con  corriente directa”
Se trata de hacer girar un motor de corriente continua en los dos sentidos posibles de giro (derecha o izquierda).
 Un motor cambia de sentido de giro cuando cambia su polaridad en su bornes (contactos) es decir cambiar el sentido de la corriente esto es gracias a la bobina que contiene que hace un efecto de electro imán y al cambiarle el eje de lado opuesto este puede generar una cantidad de voltaje dependiendo de la fuerza con que se manejé. 
 \section{Biblio}
http://motores.nichese.com/inversiongiro.html\\
http://motores.nichese.com/motor20cc.html\\
https://clr.es/blog/es/motores-corriente-continua-alterna-seleccion/\\

\end{document}