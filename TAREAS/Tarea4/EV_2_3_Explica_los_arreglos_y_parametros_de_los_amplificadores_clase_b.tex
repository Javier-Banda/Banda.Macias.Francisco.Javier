\documentclass[11pt,a4paper]{article}
\usepackage{graphicx}
\usepackage{amsmath}
\usepackage{amssymb}
\usepackage{mathrsfs}
\usepackage{cancel}

\begin{document}
\begin{center}
\textbf{TAREA4}\\
EXPLICA LOS ARREGLOS Y PARAMETROS DE LOS AMPLIFICADORES CLASE B
\end{center}

\begin{center}
Banda Macias Francisco Javier\\
08-OCT-2019\\
Universidad Politecnica de La Zona Metropolitana de Guadalajara
\end{center}


\section{Que es un amplificador clase b}
Los amplificadores de clase B se caracterizan por tener intensidad casi nula a través de sus transistores cuando no hay señal en la entrada del circuito, por lo que en reposo el consumo es casi nulo.
Se les denomina amplificador clase B, cuando el voltaje de polarización y la máxima amplitud de la señal entrante poseen valores que hacen que la corriente de salida circule durante el semi ciclo de la señal de entrada.
La característica principal de este tipo de amplificadores es el alto factor de amplificación.
\section{Ventajas}
       1-Posee bajo consumo en reposo.\\
          2-Aprovecha al máximo la Corriente entregada por la fuente.\\
            3-Intensidad casi nula cuando está en reposo.\\
El funcionamiento del amplificador de clase B tiene polarización CC cero ya que los transistores están polarizados en el corte, por lo que cada transistor solo conduce cuando la señal de entrada es mayor que la tensión del emisor base . Por lo tanto, en la entrada cero hay salida cero y no se consume energía. Esto significa que el punto Q real de un amplificador de Clase B está en la parte Vce de la línea de carga, como se muestra a continuación.


\end{document}