\documentclass[11pt,a4paper]{article}
\usepackage{graphicx}
\usepackage{amsmath}
\usepackage{amssymb}
\usepackage{mathrsfs}
\usepackage{cancel}

\begin{document}
\begin{center}
\textbf{TAREA 3}\\
\textbf{EV 2-2 Explicar Los arreglos y Parametros de los amplificadores clase A}
\end{center}

\begin{center}
Banda Macias Francisco Javier\\
\textbf{01-oct-2019}\\
\textbf{Universidad Politecnica de La Zona Metropolitana de Guadalajara}
\end{center}

\section{Amplificadores de Clase A}

Un amplificador operacional es un tipo de amplificador, un amplificador es cualquier elemento al que le pones una señal electrónica de entrada para así poder obtener una salida de mayor voltaje o corriente sobre la salida obtenida por el amplificador operacional.
El amplificador operacional tienes como símbolo un triángulo donde tiene dos entradas una es positiva y la otra negativa cuenta con una entrada de corriente o voltaje positiva y la otra entrada de voltaje o corriente negativa así vez tiene también una salida de voltaje o corriente.
Lo que hace esto es tener una ganancia alta dónde solo son útiles para circuito de retroalimentación lo que cuenta o se le llama una entrada diferencial que hace que de la entrada positiva o negativa salga la diferencia de está por ejemplo Vsalida =Ganancia ( voltaje positivo – voltaje negativo)
Es un dispositivo de acoplo directo con entrada diferencial, y un único terminal de salida. El amplificador sólo responde a la diferencia de tensión entre los dos terminales de entrada, no a su potencial común. Una señal positiva en la entrada inversora (-), produce una señal negativa a la salida, mientras que la misma señal en la entrada no inversora (+) produce una señal positiva en la salida. Con una tensión de entrada diferencial, Vd, la tensión de salida, Vo, será a Vd, donde a es la ganancia del amplificador. Ambos terminales de entrada del amplificador se utilizarán siempre independientemente de la aplicación. La señal de salida es de un sólo terminal y está referida a masa, por consiguiente, se utilizan tensiones de alimentación bipolares ( ± ).\\
\end{document}