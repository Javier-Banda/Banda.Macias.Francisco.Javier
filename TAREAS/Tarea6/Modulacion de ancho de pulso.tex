\documentclass[11pt,a4paper]{article}
\usepackage{graphicx}
\usepackage{amsmath}
\usepackage{amssymb}
\usepackage{mathrsfs}
\usepackage{cancel}

\begin{document}
\begin{center}
\textbf{TAREA6}\\ 
Diseño  de una modulacion de ancho de pulso (PWM) con amp-op y transistores 
\end{center}

\begin{center}
Banda Macias Francisco Javier\\
22-OCT-2019\\
Universidad Politecnica de La Zona Metropolitana de Guadalajara
\end{center}


\section{Que es un amplificador clase b}
“Diseñó de modulación de ancho de pulso (PWM) con AMP-OP y transistores”
Un PMW es una  Modelación de ancho de pulso ( pulse width modulation) la cual arroja una señal cuadrada que varía el tiempo que se encuentra en estado alto sin variar su frecuencia.
Es un diseño de un inversor monofásico de frecuencia variable y modulación de ancho de pulso o PWM, lo que 
significa que se puede variar el ancho del pulso alterno producido. Esto es un 
convertidor de potencia DC a potencia AC con la posibilidad de variar la frecuencia 
dentro de un amplio rango de trabajo.
La señal que emite este diseño  es una onda cuadrada ya que esto lo hace porque va disminuyendo el voltaje así mismo es como puedes regular el voltaje de algún led entre otras opciones.
El ancho de pulso no es más que el tiempo que se tiene con valor alto en el voltaje. PWM es solo una señal que sube y baja, sube y baja haciendo más corto el periodo.
Nos sirve para transferir datos por ancho de pulso asta controlar un led o un motor.
La tención media se mide Vm= Vp• Dc


\end{document}